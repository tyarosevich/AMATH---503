\documentclass{article}
\usepackage{fancyhdr}
\pagestyle{fancy}
\fancyfoot{}
\fancyfoot[C]{\thepage}
\rhead{Yarosevich - 1064712}
\usepackage{setspace}
\usepackage{siunitx} % Provides the \SI{}{} and \si{} command for typesetting SI units
\usepackage{graphicx} % Required for the inclusion of images
\usepackage{amsmath} % Required for some math elements 
\usepackage[export]{adjustbox} % loads also graphicx
\usepackage{listings}
\usepackage{setspace}
\usepackage{matlab-prettifier}
\usepackage{float}
\usepackage[most]{tcolorbox}
\usepackage{amsfonts}
\usepackage{color}
\usepackage{titlesec}
\usepackage{caption}
\usepackage{subcaption}
\usepackage{placeins}
\usepackage{bm}
\usepackage{esvect}
\newcommand{\uveci}{{\bm{\hat{\textnormal{\bfseries\i}}}}}
\newcommand{\uvecj}{{\bm{\hat{\textnormal{\bfseries\j}}}}}
\DeclareRobustCommand{\uvec}[1]{{%
  \ifcsname uvec#1\endcsname
     \csname uvec#1\endcsname
   \else
    \bm{\hat{\mathbf{#1}}}%
   \fi
}}


\newcommand{\R}{\mathbb{R}}

\usepackage{xcolor}

\DeclareCaptionFont{white}{\color{white}}
\DeclareCaptionFormat{listing}{%
  \parbox{\textwidth}{\colorbox{gray}{\parbox{\textwidth}{#1#2#3}}\vskip-4pt}}
\captionsetup[lstlisting]{format=listing,labelfont=white,textfont=white}
\lstset{frame=lrb,xleftmargin=\fboxsep,xrightmargin=-\fboxsep}
\titleformat{\section}[runin]
  {\normalfont\Large\bfseries}{\thesection}{1em}{}
\titleformat{\subsection}[runin]
  {\normalfont\large\bfseries}{\thesubsection}{1em}{}


\setlength\parindent{0pt} % Removes all indentation from paragraphs

\renewcommand{\labelenumi}{\alph{enumi}.} % Make numbering in the enumerate environment by letter rather than number (e.g. section 6)

%\usepackage{times} % Uncomment to use the Times New Roman font

%----------------------------------------------------------------------------------------
%	DOCUMENT INFORMATION
%----------------------------------------------------------------------------------------

\title{AMATH 503: Homework 5 \\Due May, 28 2019 \\ ID: 1064712} % Title

\author{Trent \textsc{Yarosevich}} % Author name

\date{\today} % Date for the report

\begin{document}
\maketitle % Insert the title, author and date
\setlength\parindent{1cm}

\begin{center}
\begin{tabular}{l r}
%Date Performed: December 1, 2017 \\ % Date the experiment was performed
Instructor: Professor Ka-Kit Tung % Instructor/supervisor
\end{tabular}
\end{center}
\doublespacing
% If you wish to include an abstract, uncomment the lines below
% \begin{abstract}
% Abstract text
% \end{abstract}

%----------------------------------------------------------------------------------------
%	SECTION 1
%----------------------------------------------------------------------------------------
\section*{\textbf{(1)}}
If we have a Bessel Equation of the form:
\begin{equation}
\begin{aligned}
(xy')' + (\lambda^2x - \frac{m^2}{x})y = 0\\
0 < x < a\\
y \text{ bounded at } x = 0, \text{ , } y(a) = 0\\
\end{aligned}
\end{equation}
We know from the notes and previous homework that this will be solved by the eigenfunctions $J_m(\lambda x)$. The eigenfunctions are derived with the Frobenius solution, and the eigenvalues are implicitly determined by from the zeros of the eigenfunctions, which are cosine-like. Given this, let the eigenfunctions be $J_m(x)$ and the eigenvalues $(\lambda_{mn} = \frac{z_{mn}}{a})$ where $z_{mn}$ are the zeros of the eigenfunction. We use the equation above and observe that this is a Sturm-Liouville  system with:
\begin{equation}
\begin{aligned}
p(x) = x\\
r(x) = x\\
q(x) = \frac{p^2}{x}\\
\end{aligned}
\end{equation}
We now consider two pairs of eigenfunctions and eigenvalues, $(J_m(x);\lambda_{mn})$ and $(J_k(x)); \lambda_{kx})$ and plug them into the Bessel's Equation, giving:
\begin{equation}
\begin{aligned}
(xJ_m(x)')' + (\lambda_{mn}^2x - \frac{p^2}{x}) = 0\\
(xJ_k(x)')' + (\lambda_{kn}^2x - \frac{p^2}{x}) = 0
\end{aligned}
\end{equation}
We then follow the logic of the general proof of S-L orthogonality by multiplying the first by $J_k(x)$ and the second by $J_m(x)$ then subtracting one from the other:
\begin{equation}
\begin{aligned}
J_k(x) (xJ_m(x)')' - J_m(x)(xJ_k(x)')' = (\lambda_{mn} - \lambda_{kn})xJ_m(x)J_k(x)\\
\end{aligned}
\end{equation}
The LHS is a derivative, so we rewrite as follows:
\begin{equation}
\begin{aligned}
\frac{d}{dx} \Big{[}J_k(x)(xJ_m(x))') - J_m(x)(J_k(x)') \Big{]} = \Big{[}(\lambda_{mn} - \lambda_{kn})xJ_m(x)J_k(x)\Big{]}\
\end{aligned}
\end{equation}
We then integrate both sides giving:
\begin{equation}
\begin{aligned}
 \Big{[}J_k(x)(xJ_m(x))') - J_m(x)(J_k(x)') \Big{]}\Big{|}_0^a = \int_0^a\Big{[}(\lambda_{mn} - \lambda_{kn})xJ_m(x)J_k(x)\Big{]}dx
\end{aligned}
\end{equation}
We then observe that, since this is a singular S-L system and $p(x) = 0$ at $x=0$ and $x=a$. In this case, $p(x) = x$ so it' a little confusing, but let's just suppose we have a dummy variable for a moment, and $p(s) = s$. Then in a singular S-L system, $p(s) = x = 0$ when $s = 0, a$. From this we can infer that the LHS must be identically zero. This gives:
\begin{equation}
 \begin{aligned}
(\lambda_{mn} - \lambda_{kn})\int_0^axJ_m(x)J_k(x)dx = 0
\end{aligned}
\end{equation}
We can now simply observe that, if $\lambda_{mn} = \lambda_{kn}$, the leading constant becomes zero, and the integral becomes:
\begin{equation}
\begin{aligned}
\int_0^ax(J_m(x))^2dx
\end{aligned}
\end{equation}
This integral is a positive constant since $x>0$ for this Bessel function, and the eigenfunction is squared. The integrand $ax(J_m(x))^2 >0$ and therefore the resulting integral will be a positive constant. \\
\\
Alternatively, if $\lambda_{mn} \neq \lambda_{kn}$, this integral must be identically zero. The resulting integral is thus:
\begin{tcolorbox}[minipage,colback=white,arc=0pt,outer arc=0pt]
\begin{equation}
\begin{aligned}
\int_0^axJ_m(x)J_k(x)dx =
 \begin{cases} 
      0 & \lambda_{mn} \neq \lambda_{kn} \\
      c & \lambda_{mn} = \lambda_{kn}
   \end{cases}
\end{aligned}
\end{equation}
Where $c>0$ is a constant.
\end{tcolorbox}
\section*{\textbf{(2)}}
\subsection*{\textbf{(a)}}
From the prompt we know that, with spherical symmetry, the 3D wave equation becomes:
\begin{equation}
\begin{aligned}
u_{tt} = \frac{c^2}{r}(ru)_{rr}
\end{aligned}
\end{equation}
Bringing the $r$ to the LHS, we can note that it is constant variable with respect to $t$, and we can write the PDE as:
\begin{equation}
\begin{aligned}
(ru)_{tt} = c^2(ru)_{rr}
\end{aligned}
\end{equation}
Now let's substitute $v= ru$ and plug the resulting equation into the D'Alembert Solution:
\begin{equation}
\begin{aligned}
v_{tt} = c^2v_{rr}\\
v(r,t) = \frac{1}{2} \Big{[}g(r - ct) + g(r + ct) + \frac{1}{2c}\int^{r+ct}_{r-ct}h(s)ds
\end{aligned}
\end{equation}
To transform $g$ back to its equivalent term in $u$:
\begin{equation}
\begin{aligned}
v(r,0) = g(r)\\
\end{aligned}
\end{equation}
And the final form of the equation is:
\begin{equation}
\begin{aligned}
u(r,0) = f(r)
\end{aligned}
\end{equation}
Thus we have:
\begin{equation}
\begin{aligned}
\frac{1}{r}v(r,0) = u(r,0)\\
\frac{1}{r}g(r) = f(r)\\
g(r) = r f(r)\\
\end{aligned}
\end{equation}
Similarly, we observe that:
\begin{equation}
\begin{aligned}
v_t = \frac{d}{dt} \Big{[}\frac{1}{2c}\int^{r+ct}_{r-ct}h(s)ds \Big{]}\\
v_t =\frac{1}{2c} \frac{d}{dt} \Big{[}H(r+ct) - H(r-ct) \Big{]}\\
v_t = \frac{1}{2c} \Big{[}h(r+ct)(c) - h(r-ct)(-c)\Big{]}\\
v_t = \frac{1}{2} \Big{[}h(r+ct) + h(r-ct)\\
v_t(r,0) = h(r)\\
ru_t(r,0) = h(r)\\
u_t(r,0) = \frac{1}{r}h(r)\\
\text{So we define some new function:}\\
rk(r) = h(r)
\end{aligned}
\end{equation}
We then substitute all the transformed terms into the solution and get:
\begin{tcolorbox}[minipage,colback=white,arc=0pt,outer arc=0pt]
\begin{equation}
\begin{aligned}
ru(r,t) = \frac{1}{2} \Big{[}(r-ct)f(r-ct) + (r+ct)f(r+ct) + \frac{1}{2c}\int_{r-ct}^{r+ct}sk(s)ds\\
u(r,t) = \frac{1}{2r} \Big{[}(r-ct)f(r-ct) + (r+ct)f(r+ct) + \frac{1}{2c}\int_{r-ct}^{r+ct}sk(s)ds\\
\end{aligned}
\end{equation}
\end{tcolorbox}

\end{document}