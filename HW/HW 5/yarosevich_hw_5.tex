\documentclass{article}
\usepackage{fancyhdr}
\pagestyle{fancy}
\fancyfoot{}
\fancyfoot[C]{\thepage}
\rhead{Yarosevich - 1064712}
\usepackage{setspace}
\usepackage{siunitx} % Provides the \SI{}{} and \si{} command for typesetting SI units
\usepackage{graphicx} % Required for the inclusion of images
\usepackage{amsmath} % Required for some math elements 
\usepackage[export]{adjustbox} % loads also graphicx
\usepackage{listings}
\usepackage{setspace}
\usepackage{matlab-prettifier}
\usepackage{float}
\usepackage[most]{tcolorbox}
\usepackage{amsfonts}
\usepackage{color}
\usepackage{titlesec}
\usepackage{caption}
\usepackage{subcaption}
\usepackage{placeins}
\usepackage{bm}
\usepackage{esvect}
\newcommand{\uveci}{{\bm{\hat{\textnormal{\bfseries\i}}}}}
\newcommand{\uvecj}{{\bm{\hat{\textnormal{\bfseries\j}}}}}
\DeclareRobustCommand{\uvec}[1]{{%
  \ifcsname uvec#1\endcsname
     \csname uvec#1\endcsname
   \else
    \bm{\hat{\mathbf{#1}}}%
   \fi
}}


\newcommand{\R}{\mathbb{R}}

\usepackage{xcolor}

\DeclareCaptionFont{white}{\color{white}}
\DeclareCaptionFormat{listing}{%
  \parbox{\textwidth}{\colorbox{gray}{\parbox{\textwidth}{#1#2#3}}\vskip-4pt}}
\captionsetup[lstlisting]{format=listing,labelfont=white,textfont=white}
\lstset{frame=lrb,xleftmargin=\fboxsep,xrightmargin=-\fboxsep}
\titleformat{\section}[runin]
  {\normalfont\Large\bfseries}{\thesection}{1em}{}
\titleformat{\subsection}[runin]
  {\normalfont\large\bfseries}{\thesubsection}{1em}{}


\setlength\parindent{0pt} % Removes all indentation from paragraphs

\renewcommand{\labelenumi}{\alph{enumi}.} % Make numbering in the enumerate environment by letter rather than number (e.g. section 6)

%\usepackage{times} % Uncomment to use the Times New Roman font

%----------------------------------------------------------------------------------------
%	DOCUMENT INFORMATION
%----------------------------------------------------------------------------------------

\title{AMATH 503: Homework 5 \\Due May, 28 2019 \\ ID: 1064712} % Title

\author{Trent \textsc{Yarosevich}} % Author name

\date{\today} % Date for the report

\begin{document}
\maketitle % Insert the title, author and date
\setlength\parindent{1cm}

\begin{center}
\begin{tabular}{l r}
%Date Performed: December 1, 2017 \\ % Date the experiment was performed
Instructor: Professor Ka-Kit Tung % Instructor/supervisor
\end{tabular}
\end{center}
\doublespacing
% If you wish to include an abstract, uncomment the lines below
% \begin{abstract}
% Abstract text
% \end{abstract}

%----------------------------------------------------------------------------------------
%	SECTION 1
%----------------------------------------------------------------------------------------
\section*{\textbf{(1)}}
If we have a Bessel Equation of the form:
\begin{equation}
\begin{aligned}
(xy')' + (\lambda^2x - \frac{m^2}{x})y = 0\\
0 < x < a\\
y \text{ bounded at } x = 0, \text{ , } y(a) = 0\\
\end{aligned}
\end{equation}
We know from the notes and previous homework that this will be solved by the eigenfunctions $J_m(\lambda x)$. The eigenfunctions are derived with the Frobenius solution, and the eigenvalues are implicitly determined by from the zeros of the eigenfunctions, which are cosine-like. Given this, let the eigenfunctions be $J_m(\lambda_{mn}x)$ and the eigenvalues $(\lambda_{mn} = \frac{z_{mn}}{a})$ where $z_{mn}$ are the zeros of the eigenfunction. We use the equation above and observe that this is a Sturm-Liouville  system with:
\begin{equation}
\begin{aligned}
p(x) = x\\
r(x) = x\\
q(x) = \frac{p^2}{x}\\
\end{aligned}
\end{equation}
We now consider two pairs of eigenfunctions and eigenvalues, $(J_m(\lambda_{mn}x);\lambda_{mn})$ and $(J_k(\lambda_{kn}x)); \lambda_{kx})$ and plug them into the Bessel's Equation, giving:
\begin{equation}
\begin{aligned}
(xJ_m(\lambda_{mn}x)')' + (\lambda_{mn}^2x - \frac{p^2}{x}) = 0\\
(xJ_k(\lambda_{mn}x)')' + (\lambda_{kn}^2x - \frac{p^2}{x}) = 0
\end{aligned}
\end{equation}
We then follow the logic of the general proof of S-L orthogonality by multiplying the first by $J_k(\lambda_{kn}x)$ and the second by $J_m(\lambda_{mn}x)$ then subtracting one from the other:
\begin{equation}
\begin{aligned}
J_k(\lambda_{kn}x) (xJ_m(\lambda_{mn}x)')' - J_m(\lambda_{mn}x)(xJ_k(\lambda_{kn}x)')' = (\lambda_{mn} - \lambda_{kn})xJ_m(\lambda_{mn}x)J_k(\lambda_{kn}x)\\
\end{aligned}
\end{equation}
The LHS is a derivative, so we rewrite as follows:
\begin{equation}
\begin{aligned}
\frac{d}{dx} \Big{[}J_k(\lambda_{kn}x)(xJ_m(\lambda_{mn}x))') - J_m(\lambda_{mn}x)(J_k(\lambda_{kn}x)') \Big{]} = \Big{[}(\lambda_{mn} - \lambda_{kn})xJ_m(x)J_k(x)\Big{]}\
\end{aligned}
\end{equation}
We then integrate both sides giving:
\begin{equation}
\begin{aligned}
 \Big{[}J_k(\lambda_{kn}x)(xJ_m(\lambda_{mn}x))') - J_m(\lambda_{mn}x)(J_k(\lambda_{kn}x)') \Big{]}\Big{|}_0^a = \int_0^a\Big{[}(\lambda_{mn} - \lambda_{kn})xJ_m(\lambda_{mn}x)J_k(\lambda_{kn}x)\Big{]}dx
\end{aligned}
\end{equation}
We then observe that, since this is a singular S-L system and $p(x) = 0$ at $x=0$ and $x=a$. In this case, $p(x) = x$ so it' a little confusing, but let's just suppose we have a dummy variable for a moment, and $p(s) = s$. Then in a singular S-L system, $p(s) = x = 0$ when $s = 0, a$. From this we can infer that the LHS must be identically zero. This gives:
\begin{equation}
 \begin{aligned}
(\lambda_{mn} - \lambda_{kn})\int_0^axJ_m(\lambda_{mn}x)J_k(\lambda_{kn}x)dx = 0
\end{aligned}
\end{equation}
We can now simply observe that, if $\lambda_{mn} = \lambda_{kn}$, the leading constant becomes zero, and the integral becomes:
\begin{equation}
\begin{aligned}
\int_0^ax(J_m(\lambda_{mn}x))^2dx
\end{aligned}
\end{equation}
This integral is a positive constant since $x>0$ for this Bessel function, and the eigenfunction is squared. The integral is $ax(J_m(a\lambda_{mn}x))^2 >0$ and therefore a positive constant. \\
\\
Alternatively, if $\lambda_{mn} \neq \lambda_{kn}$, this integral must be identically zero. The resulting integral is thus:
\begin{tcolorbox}[minipage,colback=white,arc=0pt,outer arc=0pt]
\begin{equation}
\begin{aligned}
\int_0^axJ_m(\lambda_{mn}x)J_k(\lambda_{kn}x)dx =
 \begin{cases} 
      0, & \lambda_{mn} \neq \lambda_{kn} \\
      c, & \lambda_{mn} = \lambda_{kn}
   \end{cases}
\end{aligned}
\end{equation}
Where $c>0$ is a constant.
\end{tcolorbox}
\section*{\textbf{(2)}}
\subsection*{\textbf{(a)}}
From the prompt we know that, with spherical symmetry, the 3D wave equation becomes:
\begin{equation}
\begin{aligned}
u_{tt} = \frac{c^2}{r}(ru)_{rr}
\end{aligned}
\end{equation}
Bringing the $r$ to the LHS, we can note that it is constant variable with respect to $t$, and we can write the PDE as:
\begin{equation}
\begin{aligned}
(ru)_{tt} = c^2(ru)_{rr}
\end{aligned}
\end{equation}
Now let's substitute $v= ru$ and plug the resulting equation into the D'Alembert Solution:
\begin{equation}
\begin{aligned}
v_{tt} = c^2v_{rr}\\
v(r,t) = \frac{1}{2} \Big{[}g(r - ct) + g(r + ct) + \frac{1}{2c}\int^{r+ct}_{r-ct}h(s)ds
\end{aligned}
\end{equation}
To transform $g$ back to its equivalent term in $u$:
\begin{equation}
\begin{aligned}
v(r,0) = g(r)\\
\end{aligned}
\end{equation}
And the final form of the equation is:
\begin{equation}
\begin{aligned}
u(r,0) = f(r)
\end{aligned}
\end{equation}
Thus we have:
\begin{equation}
\begin{aligned}
\frac{1}{r}v(r,0) = u(r,0)\\
\frac{1}{r}g(r) = f(r)\\
g(r) = r f(r)\\
\end{aligned}
\end{equation}
Similarly, we observe that:
\begin{equation}
\begin{aligned}
v_t = \frac{d}{dt} \Big{[}\frac{1}{2c}\int^{r+ct}_{r-ct}h(s)ds \Big{]}\\
v_t =\frac{1}{2c} \frac{d}{dt} \Big{[}H(r+ct) - H(r-ct) \Big{]}\\
v_t = \frac{1}{2c} \Big{[}h(r+ct)(c) - h(r-ct)(-c)\Big{]}\\
v_t = \frac{1}{2} \Big{[}h(r+ct) + h(r-ct)\\
v_t(r,0) = h(r)\\
ru_t(r,0) = h(r)\\
u_t(r,0) = \frac{1}{r}h(r)\\
\text{So we define some new function:}\\
u_t(r,0) = k(r)\\
rk(r) = h(r)
\end{aligned}
\end{equation}
We then substitute all the transformed terms into the solution and replace the $r$ terms with suitable $r\pm ct$ or $s$ and get:
\begin{tcolorbox}[minipage,colback=white,arc=0pt,outer arc=0pt]
\begin{equation}
\begin{aligned}
ru(r,t) = \frac{1}{2} \Big{[}(r-ct)f(r-ct) + (r+ct)f(r+ct) + \frac{1}{2c}\int_{r-ct}^{r+ct}sk(s)ds\\
u(r,t) = \frac{1}{2r} \Big{[}(r-ct)f(r-ct) + (r+ct)f(r+ct) + \frac{1}{2c}\int_{r-ct}^{r+ct}sk(s)ds\\
\end{aligned}
\end{equation}
\end{tcolorbox}
\section*{\textbf{(3)}}
\subsection*{\textbf{(a)}}
In the steady state solution $u_{tt} = 0$ thus:
\begin{equation}
\begin{aligned}
u_{xx} = - 1\\
u_x = -x + c\\
u = \frac{-x^2}{2} + cx + d\\
u(0)= 0 = d\\
u(1) = 0 = -\frac{1}{2} + c\\
c = \frac{1}{2}
\end{aligned}
\end{equation}
So the steady state solution is:
\begin{tcolorbox}[minipage,colback=white,arc=0pt,outer arc=0pt]
\begin{equation}
\begin{aligned}
u(x,t)  = \frac{-x^2}{2} + \frac{x}{2}
\end{aligned}
\end{equation}
\end{tcolorbox}
Suppose the transient solution is some $v = u - u_{\text{steady}}$. If we plug this into the PDE we get:
\begin{equation}
\begin{aligned}
v_{tt} - 0 = v_{xx} - \frac{d^2}{dx^2}(\frac{-x^2}{2} + \frac{x}{2}) + 1\\
v_{tt} = v_{xx}
\end{aligned}
\end{equation}
By separation of variables let $v=T(t)X(x)$. As we've seen countless times in the class thus far, this yields the solution:
\begin{equation}
\begin{aligned}
v(x,t) = \sum_{n=1}^{\infty} \Big{[}A_nsin(\lambda_n t) + B_ncos(\lambda_n t) \Big{]}sin(\frac{n\pi x}{L})\\
L = 1\\
\end{aligned}
\end{equation}
Now we apply the BCs noting that $u(x,t) = v(x,t) - \frac{x^2}{2} + \frac{x}{2}$\\
\begin{equation}
\begin{aligned}
u(x,0) = 0 = \sum_{n=1}^{\infty}B_nsin(n\pi x)- \frac{x^2}{2} + \frac{x}{2}\\
\sum_{n=1}^{\infty}B_nsin(n\pi x) = \frac{x^2}{2} - \frac{x}{2}\\
\end{aligned}
\end{equation}
This is a sine series and the coefficient $B_n$ s given:
\begin{equation}
\begin{aligned}
B_n = \int_0^1x^2sin(n\pi x)dx - \int_0^1xsin(n\pi x)dx\\
\end{aligned}
\end{equation}
Proceeding one integral at a time:
\begin{equation}
\begin{aligned}
\int_0^1x^2sin(n\pi x)dx = \Big{[}\frac{-x^2cos(n\pi x)}{n\pi} \Big{]}\Big{|}_0^1 + 2\int_0^1\frac{xcos(n\pi x)}{n\pi}dx\\
\int_0^1\frac{xcos(n\pi x)}{n\pi}dx = \Big{[}\frac{xsin}{(n^2\pi^2} \Big{]}\Big{|}_0^1 - \int_0^1\frac{sin(n\pi x)}{n^2\pi^2}dx\\
\int_0^1\frac{sin(n\pi x)}{n^2\pi^2}dx = \Big{[}\frac{cos(n\pi x)}{n^3\pi^3}\Big{]}\Big{|}_0^1
\end{aligned}
\end{equation}
Putting these evaluations together we have:
\begin{equation}
\begin{aligned}
 \Big{[}\frac{-x^2cos(n\pi x)}{n\pi} \Big{]}\Big{|}_0^1 & + 2 \Big{[}\Big{[}\frac{xsin}{(n^2\pi^2} \Big{]}\Big{|}_0^1 + \Big{[}\frac{cos(n\pi x)}{n^3\pi^3}\Big{]}\Big{|}_0^1 \Big{]}\\
 &= \frac{-(-1)^n}{n\pi} + 2 \Big{[}\frac{(-1)^n)}{n^3\pi^3} - \frac{1}{n^3\pi^3} \Big{]}
\end{aligned}
\end{equation}
Now the other integral:
\begin{equation}
\begin{aligned}
\int_0^1xsin(n\pi x)dx = \Big{[}\frac{-x cos(n\pi x)}{n\pi} \Big{]}\Big{|}_0^1 - \int_0^1\frac{cos(n\pi x)}{n\pi}dx\\
\int_0^1\frac{cos(n\pi x)}{n\pi}dx = \Big{[} \frac{sin(n\pi x)}{n^2\pi^2} \Big{]} \Big{|}_0^1 = 0\\
\int_0^1xsin(n\pi x)dx = \frac{-(-1)^n}{n\pi}
\end{aligned}
\end{equation}
Combining the two terms:
\begin{equation}
\begin{aligned}
B_n = \frac{-(-1)^n n^2\pi^2 + 2(-1)^n - 2 + (-1)^nn^2\pi^2}{n^3\pi^3}\\
B_n = \frac{2( (-1)^n - 1)}{n^3\pi^3}
\end{aligned}
\end{equation}
Now we define $n = 1,3,5...$ giving:
\begin{equation}
\begin{aligned}
B_n = \frac{-4}{n^3\pi^3}
\end{aligned}
\end{equation}
Applying the other initial condition we take the derivative:
\begin{equation}
\begin{aligned}
u_t(x,t) = \sum_{n=1}^{\infty} \Big{[}n\pi  A_ncos(n\pi t) - n\pi B_nsin(n\pi t) \Big{]}sin(n\pi x)\\
u_t(x,0) = \sum_{n=1}^{\infty}n\pi A_n sin(n\pi x) = 0\\
A_n = 0
\end{aligned}
\end{equation}
Thus the transient solution and complete solution are:
\begin{tcolorbox}[minipage,colback=white,arc=0pt,outer arc=0pt]
\begin{equation}
\begin{aligned}
v(x,t) = \sum_{n=1}^{\infty}\frac{-4}{n^3\pi^3}cos(n\pi t)sin(n\pi x)\\
u(x,t) = \sum_{n=1}^{\infty}\frac{-4}{n^3\pi^3}cos(n\pi t)sin(n\pi x) - \frac{x^2}{2} + \frac{x}{2}\\
n = 1,3,5...
\end{aligned}
\end{equation}
\end{tcolorbox}
\subsection*{\textbf{(b)}}
First we determine the eigenfunctions and eigenvalues that will satisfy the boundary conditions. They are homogeneous dirchelet, and so the eigenfunction is $sin(n\pi x)$. We thus represent the homogeneous form of the PDE as:
\begin{equation}
\begin{aligned}
u(x,t) = \sum_{n=1}^{\infty}T(t)sin(n\pi x)
\end{aligned}
\end{equation}
We then represent the forcing term with the eigenfunction/values:
\begin{equation}
\begin{aligned}
f(x,t) = 1 = \sum_{n=1}^{\infty}f_n sin(n\pi x)
\end{aligned}
\end{equation}
Next we plug this into the PDE. 
\begin{equation}
\begin{aligned}
\sum_{n=1}^{\infty} T_n''(t)X(x) - X_n''(x)T_n(t) = \sum_{n=1}^{\infty}f_n sin(n\pi x)\\
T_n''(t)sin(n\pi x) + (n\pi)^2 sin(n\pi x)T_n(t) = f_nsin(n\pi x)\\
T_n''(t) + (n\pi)^2T_n(t) = f_n
\end{aligned}
\end{equation}
To proceed we'll need $f_n$, which is the coefficient for the sine series for $f(x) = 1$:
\begin{equation}
\begin{aligned}
A_n  = 2\int_0^1sin(n\pi x) = 2\Big{[}\frac{-cos(n\pi x)}{n\pi}\Big{]}\Big{|}_0^1\\
A_n = 2\frac{1-(-1)^n}{n\pi}\\
A_n = 0 \text{ for even } n,\\
A_n = \frac{4}{n\pi} \text{ for odd } n
\end{aligned}
\end{equation}
We thus have two scenarios:
\begin{equation}
\begin{aligned}
n = 1, 3, 5... &\to T_n''(t) + (n\pi)^2T_n(t) = \frac{4}{n\pi}\\
n = 2, 4, 6... &\to T_n''(t) + (n\pi)^2T_n(t) = 0
\end{aligned}
\end{equation}
In the even case, we know from major precedent that the solution is sines and cosines:
\begin{equation}
\begin{aligned}
n = 2,4,6...\\
T_n(t) = A_n sin(n\pi t) + B_n cos(n\pi t)
\end{aligned}
\end{equation}
For the odd $n$ case we need the homogeneous solution plus the particular. We already have the former, and so to find the particular solution we guess a solution in the form of the forcing term, i.e. a constant $k$ and plug it into the PDE:
\begin{equation}
\begin{aligned}
T(t)_{\text{particular}} = k\\
k(n\pi)^2= \frac{4}{n\pi}\\
k = \frac{4}{(n\pi)^3}
\end{aligned}
\end{equation}
Thus the complete solution for $T(t)$ for odd $n$ is:
\begin{equation}
\begin{aligned}
T_n(t) = \sum_{n=1}^{\infty}A_n sin(n\pi t) + B_n cos(n\pi t) + \frac{4}{(n\pi)^3}
\end{aligned}
\end{equation}
Now, I'm not sure how to articulate this accurately, but I think the $n=2,4,6..$ case is not a part of the solution, since it actually doesn't satisfy the PDE. I speculate that this is because $f_n$ can be defined with strictly odd $n$. We will need to solve the homogeneous case, but the general solution is of the form:
\begin{tcolorbox}[minipage,colback=white,arc=0pt,outer arc=0pt]
\begin{equation}
\begin{aligned}
u(x,t) = \sum_{n=1}^{\infty} \Big{[} A_n sin(n\pi t) + B_n cos(n\pi t) +\frac{4}{(n\pi)^3} \Big{]}sin(n\pi x)\\
n = 1,3,5...
\end{aligned}
\end{equation}
\end{tcolorbox}
Applying the Initial Conditions:
\begin{equation}
\begin{aligned}
u(x,0) = \sum_{n=1}^{\infty} \Big{[}B_n + \frac{4}{(n\pi)^3}\Big{]}sin(n\pi x) = 0\\
B_n = -\frac{4}{(n\pi)^3}\\
u_t(x,t) = \sum_{n=1}^{\infty} \Big{[} n\pi A_n cos(n\pi t) - n\pi B_n sin(n\pi t) \Big{]}sin(n\pi x)\\
u_t(x,0) = \sum_{n=1}^{\infty}\Big{[}n\pi A_n \Big{]}sin(n\pi x) = 0\\
A_n = 0
\end{aligned}
\end{equation}
This gives a final solution of:
\begin{tcolorbox}[minipage,colback=white,arc=0pt,outer arc=0pt]
\begin{equation}
\begin{aligned}
u(x,t) = \sum_{n=1}^{\infty} \Big{[} \frac{4}{(n\pi)^3}( 1- cos(n\pi t)) \Big{]}sin(n\pi x)\\
n = 1,3,5...
\end{aligned}
\end{equation}
\end{tcolorbox}
If we note from \textbf{(22)} above that 
\begin{equation}
\begin{aligned}
n = 1,3,5..\\
\sum_{n=1}^{\infty}\frac{-4}{(n\pi)^3}sin(n\pi x) = \frac{x^2}{2} - \frac{x}{2}\\
\sum_{n=1}^{\infty}\frac{4}{(n\pi)^3}sin(n\pi x) = -\frac{x^2}{2} +\frac{x}{2}\\
\end{aligned}
\end{equation}
We can see that the extra series term in \textbf{(42)} is equivalent to the non-series term in \textbf{(30)} and we have the same answer.
\end{document}