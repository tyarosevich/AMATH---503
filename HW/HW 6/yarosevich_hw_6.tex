\documentclass{article}
\usepackage{fancyhdr}
\pagestyle{fancy}
\fancyfoot{}
\fancyfoot[C]{\thepage}
\rhead{Yarosevich - 1064712}
\usepackage{setspace}
\usepackage{siunitx} % Provides the \SI{}{} and \si{} command for typesetting SI units
\usepackage{graphicx} % Required for the inclusion of images
\usepackage{amsmath} % Required for some math elements 
\usepackage[export]{adjustbox} % loads also graphicx
\usepackage{listings}
\usepackage{setspace}
\usepackage{matlab-prettifier}
\usepackage{physics}
\usepackage{float}
\usepackage[most]{tcolorbox}
\usepackage{amsfonts}
\usepackage{color}
\usepackage{titlesec}
\usepackage{caption}
\usepackage{subcaption}
\usepackage{placeins}
\usepackage{bm}
\usepackage{esvect}
\newcommand{\uveci}{{\bm{\hat{\textnormal{\bfseries\i}}}}}
\newcommand{\uvecj}{{\bm{\hat{\textnormal{\bfseries\j}}}}}
\DeclareRobustCommand{\uvec}[1]{{%
  \ifcsname uvec#1\endcsname
     \csname uvec#1\endcsname
   \else
    \bm{\hat{\mathbf{#1}}}%
   \fi
}}


\newcommand{\R}{\mathbb{R}}

\usepackage{xcolor}

\DeclareCaptionFont{white}{\color{white}}
\DeclareCaptionFormat{listing}{%
  \parbox{\textwidth}{\colorbox{gray}{\parbox{\textwidth}{#1#2#3}}\vskip-4pt}}
\captionsetup[lstlisting]{format=listing,labelfont=white,textfont=white}
\lstset{frame=lrb,xleftmargin=\fboxsep,xrightmargin=-\fboxsep}
\titleformat{\section}[runin]
  {\normalfont\Large\bfseries}{\thesection}{1em}{}
\titleformat{\subsection}[runin]
  {\normalfont\large\bfseries}{\thesubsection}{1em}{}


\setlength\parindent{0pt} % Removes all indentation from paragraphs

\renewcommand{\labelenumi}{\alph{enumi}.} % Make numbering in the enumerate environment by letter rather than number (e.g. section 6)

%\usepackage{times} % Uncomment to use the Times New Roman font

%----------------------------------------------------------------------------------------
%	DOCUMENT INFORMATION
%----------------------------------------------------------------------------------------

\title{AMATH 503: Homework 5 \\Due May, 28 2019 \\ ID: 1064712} % Title

\author{Trent \textsc{Yarosevich}} % Author name

\date{\today} % Date for the report

\begin{document}
\maketitle % Insert the title, author and date
\setlength\parindent{1cm}

\begin{center}
\begin{tabular}{l r}
%Date Performed: December 1, 2017 \\ % Date the experiment was performed
Instructor: Professor Ka-Kit Tung % Instructor/supervisor
\end{tabular}
\end{center}
\doublespacing
% If you wish to include an abstract, uncomment the lines below
% \begin{abstract}
% Abstract text
% \end{abstract}

%----------------------------------------------------------------------------------------
%	SECTION 1
%----------------------------------------------------------------------------------------
\section*{\textbf{(1)}}
\subsection*{\textbf{(a)}}
Given the solution to the Green's Function for the Heat Equation in 1-D stated on page 228 of the notes, we can define the semi-infinite domain as $0<x<\infty$, and then subtract the same solution defined at the location $x = -\xi$ to satisfy the boundary condition $G=0$ at $x=0$. This gives the Green's Function:
\begin{equation}
\begin{aligned}
G(x, t;\xi, \tau) &= \frac{1}{\sqrt{4\pi D(t-\tau)}}\Big{[} e^{-\frac{(x-\xi)^2}{4D(t-\tau)}} -e^{-\frac{(x+\xi)^2}{4D(t-\tau)}}\Big{]}
\end{aligned}
\end{equation}
To show this satisfies the BC, we can simply observe that if $x=0$, the two exponential terms are equivalent since the only difference is the $\pm\xi$ term, which is squared anyway. The other boundary condition is still satisfied since both of these exponentials are raised to a strictly negative power increasing with x, and will go to zero as $x\to\infty$.
\subsection*{\textbf{(b)}}
To satisfy a boundary condition of $\pdv{}{x}G$ at $x=0$, we can observe that if we add a solution mirrored across the $G$ axis, the two resulting normal distributions will have equal slope, but opposite in sign, at $x=0$:
\begin{equation}
\begin{aligned}
G(x, t;\xi, \tau) &= \frac{1}{\sqrt{4\pi D(t-\tau)}}\Big{[} e^{-\frac{(x-\xi)^2}{4D(t-\tau)}}+e^{-\frac{(x+\xi)^2}{4D(t-\tau)}}\Big{]}
\end{aligned}
\end{equation}
We can show this satisfies the BC at $x=0$ by taking the derivatives of the exponentials:
\begin{equation}
\begin{aligned}
\pdv{}{x}(e^{-\frac{(x-\xi)^2}{4 D(t-\tau)}}) = \frac{-2(x-\xi)}{4D(t-\tau)}e^{-\frac{(x-\xi)^2}{4D(t-\tau)}}\\
\pdv{}{x}(e^{-\frac{(x+\xi)^2}{4D(t-\tau)}}) = \frac{-2(x+\xi)}{4D(t-\tau)}e^{-\frac{(x+\xi)^2}{4D(t-\tau)}}\\
\end{aligned}
\end{equation}
When $x=0$, again the two terms cancel since the sign on the $-\xi$ carries out, and the $\pm\xi$ in the exponents are equivalent because they are squared:
\begin{equation}
\begin{aligned}
 \frac{2(\xi)}{4D(t-\tau)}e^{-\frac{(-\xi)^2}{4D(t-\tau)}} - \frac{2(\xi)}{4D(t-\tau)}e^{-\frac{(\xi)^2}{4D(t-\tau)}}=0\\
\end{aligned}
\end{equation}
Again, the BC of $G=0$ as $x\to\infty$ is still satisfied since both normal distributions are raised to strictly negative exponent increasing with $x$.
\section*{\textbf{(2)}}
\subsection*{\textbf{(a)}}
We want to show that 
\begin{equation}
\begin{aligned}
(\pdv[2]{}{t} - c^2\pdv[2]{}{x})u =\delta(x-\xi)\delta(t-\tau)\\
u = 0 \text{ , }\pdv{}{t}u = 0 \text{ at } t=0\\
\end{aligned}
\end{equation}
is equivalent to:
\begin{equation}
\begin{aligned}
(\pdv[2]{}{t} - c^2\pdv[2]{}{x})u = 0\\
u = 0 \text{ and } \pdv{}{t}u = \delta(x-\xi) \text{ at } t=\tau
\end{aligned}
\end{equation}
To do so, we make some observations: \\
\\
(1): On account of the time delta function, by definition the RHS of the non-homogeneous equation is zero at all $t$ except for $t=\tau$. Also, because the initial condition is $u=0$ and $\pdv{}{t}u = 0$ at $=0$, we know that nothing happens until $t=\tau$. \\
\\
(2): We also know that when $t>\tau$, the RHS is again zero because of the delta function. So the original equation is actually homogeneous everywhere except at $t=\tau$.\\
\\
(3): So the question is, what happens at $t=\tau$ in the non-homogeneous PDE? We need this information to define our second initial condition as well as to ascertain whether there is a jump in $u$ at $t=\tau$, which would make this task impossible. To do this we integrate the non-homogeneous PDE with respect to time across an arbitrarily small domain around $t=\tau$:
\begin{equation}
\begin{aligned}
\int_{\tau -}^{\tau +}\Big{[} (\pdv[2]{}{t} - c^2\pdv[2]{}{x})u \Big{]}dt = \int_{\tau -}^{\tau+}\delta(x-\xi)\delta (t-\tau)dt\\
\int_{\tau -}^{\tau +}\pdv[2]{}{t}u dt - \int_{\tau -}^{\tau +}c^2\pdv[2]{}{x}u dt = \delta (x-\xi)\int_{\tau -}^{\tau+}\delta (t-\tau)dt
\end{aligned}
\end{equation}
Now a few observations. We know that when $t<\tau$ that $u=0$, so the first integral is simply equivalent to $\pdv{}{t}u$ evaluated at $\tau+$. Integrating the spatial second derivative across an arbitrarily small time span must be zero. If this weren't the case, $u$ itself would be a delta function in time, and the second derivative of $u$ in the original PDE doesn't indicate any huge singularity like that on the RHS. Lastly, the RHS is equal to $\delta(x-\xi)$ since the derivative across the zero of a delta function is by definition equal to 1. This means that one of the initial conditions of the homogeneous equation must take the following into account at $t=\tau$:
\begin{equation}
\begin{aligned}
\pdv{}{t} \Big{|}_{\tau +} = \delta(x-\xi)
\end{aligned}
\end{equation}
Lastly, we can observe that if there was a jump at $t=\tau$, $\pdv{}{t}u$ at that precise point would be very large, i.e. it would itself be equal to some kind of $t$ delta function. This isn't the case, and so we know there is no jump there and thus the other initial condition must be $u=0$ at $t=\tau$, and we have achieved our objective, as these are the "initial conditions" the prompt gives for the homogeneous, fundamental problem.
\subsection*{\textbf{(b)}}
In accordance with the D'Alembert solution we assume a solution with the form of two travelling waves moving with speed $c$ in relation to time $t-\tau$.
\begin{equation}
\begin{aligned}
u = R(x- c(t-\tau)) + L(x + c(t-\tau))
\end{aligned}
\end{equation}
Applying the first initial condition $u=0$ at $t=\tau$ we have:
\begin{equation}
\begin{aligned}
u = R(x) + L(x) = 0\\
L(x) = -R(x)\\
u = R(x + c(t-\tau) - R(x - c(t - \tau))
\end{aligned}
\end{equation}
To plug in the second initial condition, we need to take the derivative of $u$ in time and then plug in the initial condition of the derivative at $t=\tau$:
\begin{equation}
\begin{aligned}
u_t = -cR'(x+c(t-\tau)) - cR'(x-c(t-\tau))\\
u_t(t=\tau) = -2cR'(x) = \delta(x-\xi)
\end{aligned}
\end{equation}
To find $R$ we integrate, introducing a dummy variable. Note that we are integrating from $-\infty$ to $x$ since when $x>\xi$ we won't be picking up anything because we are integrating a delta function. Thus we have:
\begin{equation}
\begin{aligned}
\int R' = -\frac{1}{2c}\int \delta(x-\xi)dx\\
R = -\frac{1}{2c}\int_{-\infty}^x \delta(\bar{x}-\xi)d\bar{x}\\
\end{aligned}
\end{equation}
The integral of a delta function is the Heavyside step function, thus we have the following for $R$:
\begin{equation}
\begin{aligned}
R = -\frac{1}{2c}H(x-\xi) = 
 \begin{cases} 
      0, & x < \xi \\
      \frac{-1}{2c} &  x > \xi
   \end{cases}
\end{aligned}
\end{equation}
Thus the solution to the PDE with both initial conditions applied is as follows:
\begin{tcolorbox}[minipage,colback=white,arc=0pt,outer arc=0pt]
\begin{equation}
\begin{aligned}
u(x,t) = \frac{1}{2c} \Big{[}H(x - \xi + c(t-\tau)) - H(x- \xi - c(t-\tau)) \Big{]}
\end{aligned}
\end{equation}
\end{tcolorbox}
\subsection*{\textbf{(c)}}
To reconstruct a solution we integrate the fundamental problem multiplied by the forcing term evaluated at $\tau$ and $\xi$ across space and time to get the particular solution, noting that the evaulation of $\tau$ stops at $t$ because when $\tau > t$ the Green's Function is equal to zero anyway (I think?). 
\begin{equation}
\begin{aligned}
u_p = \frac{1}{2c}\int_0^t d\tau \int_{-\infty}^\infty \Big{[}H(x - \xi + c(t-\tau)) - H(x- \xi - c(t-\tau)) \Big{]} Q(\xi,\tau)d\xi
\end{aligned}
\end{equation}
Because the Heavyside functions describe an expanding box starting at $x=\xi$ and then expanding to the left and right as time increases, it is easy to simplify the integral to simply evaluate 1 across the domain, thus giving the area. Since we've pulled out the $\frac{1}{2c}$, the height of the box is one and it's width extends from $x-\xi - c(t-\tau)$ to $x-\xi + c(t-\tau)$. The $\xi$ can be dropped from the bounds since it would fall out in the evaluation of the integral anyway. We are left with:
\begin{tcolorbox}[minipage,colback=white,arc=0pt,outer arc=0pt]
\begin{equation}
\begin{aligned}
u_p(x,t) = \frac{1}{2c}\int_0^td\tau\int_{x-c(t-\tau)}^{x+c(t-\tau)}Q(\xi,\tau)d\xi
\end{aligned}
\end{equation}
\end{tcolorbox}
Lastly, we need the homogeneous solution to the PDE, though it seems obvious to me it must be trivial since we have zero initial condition, zero initial velocity and a zero boundary condition. Using d'Alembert's solution I'll use the variable $v$ to distinguish it from the forced PDE: $v_{tt} -c^2v_{xx} = 0$ with $v(x,0) = 0$ and $\frac{d}{dt}v(x,0) = 0$.
\begin{equation}
\begin{aligned}
v(x,t) = R(x-ct) + L(x+ct)\\
v(x,0) = 0\\
v(x,t) = R(x-ct) - R(x+ct)\\
v_t(x,t) = -cR'(x-ct) - cR'(x+ct)\\
v_t(x,0) = 0 = -2cR'(x)
\end{aligned}
\end{equation}
Integrating:
\begin{equation}
\begin{aligned}
-2cR(x) = K\\
R(x) = K\\
v(x,t) = K - K = 0\\
\end{aligned}
\end{equation}
Thus the solution should just be the above particular solution by itself.
\end{document}