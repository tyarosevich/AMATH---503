\documentclass{article}
\usepackage{fancyhdr}
\pagestyle{fancy}
\fancyfoot{}
\fancyfoot[C]{\thepage}
\rhead{Yarosevich - 1064712}
\usepackage{setspace}
\usepackage{siunitx} % Provides the \SI{}{} and \si{} command for typesetting SI units
\usepackage{graphicx} % Required for the inclusion of images
\usepackage{amsmath} % Required for some math elements 
\usepackage[export]{adjustbox} % loads also graphicx
\usepackage{listings}
\usepackage{setspace}
\usepackage{matlab-prettifier}
\usepackage{physics}
\usepackage{float}
\usepackage[most]{tcolorbox}
\usepackage{amsfonts}
\usepackage{color}
\usepackage{titlesec}
\usepackage{caption}
\usepackage{subcaption}
\usepackage{placeins}
\usepackage{bm}
\usepackage{esvect}
\newcommand{\uveci}{{\bm{\hat{\textnormal{\bfseries\i}}}}}
\newcommand{\uvecj}{{\bm{\hat{\textnormal{\bfseries\j}}}}}
\DeclareRobustCommand{\uvec}[1]{{%
  \ifcsname uvec#1\endcsname
     \csname uvec#1\endcsname
   \else
    \bm{\hat{\mathbf{#1}}}%
   \fi
}}


\newcommand{\R}{\mathbb{R}}

\usepackage{xcolor}

\DeclareCaptionFont{white}{\color{white}}
\DeclareCaptionFormat{listing}{%
  \parbox{\textwidth}{\colorbox{gray}{\parbox{\textwidth}{#1#2#3}}\vskip-4pt}}
\captionsetup[lstlisting]{format=listing,labelfont=white,textfont=white}
\lstset{frame=lrb,xleftmargin=\fboxsep,xrightmargin=-\fboxsep}
\titleformat{\section}[runin]
  {\normalfont\Large\bfseries}{\thesection}{1em}{}
\titleformat{\subsection}[runin]
  {\normalfont\large\bfseries}{\thesubsection}{1em}{}


\setlength\parindent{0pt} % Removes all indentation from paragraphs

\renewcommand{\labelenumi}{\alph{enumi}.} % Make numbering in the enumerate environment by letter rather than number (e.g. section 6)

%\usepackage{times} % Uncomment to use the Times New Roman font

%----------------------------------------------------------------------------------------
%	DOCUMENT INFORMATION
%----------------------------------------------------------------------------------------

\title{AMATH 503: Homework 5 \\Due May, 28 2019 \\ ID: 1064712} % Title

\author{Trent \textsc{Yarosevich}} % Author name

\date{\today} % Date for the report

\begin{document}
\maketitle % Insert the title, author and date
\setlength\parindent{1cm}

\begin{center}
\begin{tabular}{l r}
%Date Performed: December 1, 2017 \\ % Date the experiment was performed
Instructor: Professor Ka-Kit Tung % Instructor/supervisor
\end{tabular}
\end{center}
\doublespacing
% If you wish to include an abstract, uncomment the lines below
% \begin{abstract}
% Abstract text
% \end{abstract}

%----------------------------------------------------------------------------------------
%	SECTION 1
%----------------------------------------------------------------------------------------
\section*{\textbf{(1)}}
\subsection*{\textbf{(a)}}
Given the solution to the Green's Function for the Heat Equation in 1-D stated on page 228 of the notes, we can define the semi-infinite domain as $0<x<\infty$, and then subtract the same solution defined at the location $x = -\xi$ to satisfy the boundary condition $G=0$ at $x=0$. This gives the Green's Function:
\begin{equation}
\begin{aligned}
G(x, t;\xi, \tau) &= \frac{1}{\sqrt{4\pi D(t-\tau)}}\Big{[} e^{-\frac{(x-\xi)^2}{4D(t-\tau)}} -e^{-\frac{(x+\xi)^2}{4D(t-\tau)}}\Big{]}
\end{aligned}
\end{equation}
To show this satisfies the BC, we can simply observe that if $x=0$, the two exponential terms are equivalent since the only difference is the $\pm\xi$ term, which is squared anyway. The other boundary condition is still satisfied since both of these exponentials have negative terms, and will go to zero as $x\to\infty$.
\subsection*{\textbf{(b)}}
To satisfy a boundary condition of $\pdv{}{x}G$ at $x=0$, we can observe that if we add a solution mirrored across the $G$ axis, the two resulting normal distributions will have equal slope, but opposite in sign, at $x=0$:
\begin{equation}
\begin{aligned}
G(x, t;\xi, \tau) &= \frac{1}{\sqrt{4\pi D(t-\tau)}}\Big{[} e^{-\frac{(x-\xi)^2}{4D(t-\tau)}}+e^{-\frac{(x+\xi)^2}{4D(t-\tau)}}\Big{]}
\end{aligned}
\end{equation}
We can show this satisfies the BC at $x=0$ by taking the derivatives of the exponentials:
\begin{equation}
\begin{aligned}
\pdv{}{x}(e^{-\frac{(x-\xi)^2}{4 D(t-\tau)}}) = \frac{-2(x-\xi)}{4D(t-\tau)}e^{-\frac{(x-\xi)^2}{4D(t-\tau)}}\\
\pdv{}{x}(e^{-\frac{(x+\xi)^2}{4D(t-\tau)}}) = \frac{-2(x+\xi)}{4D(t-\tau)}e^{-\frac{(x+\xi)^2}{4D(t-\tau)}}\\
\end{aligned}
\end{equation}
When $x=0$, again the two terms cancel since the sign on the $-\xi$ carries out, and the $\pm\xi$ in the exponents are equivalent because they are squared:
\begin{equation}
\begin{aligned}
 \frac{2(\xi)}{4D(t-\tau)}e^{-\frac{(-\xi)^2}{4D(t-\tau)}} - \frac{2(\xi)}{4D(t-\tau)}e^{-\frac{(\xi)^2}{4D(t-\tau)}}=0\\
\end{aligned}
\end{equation}
Again, the BC of $G=0$ as $x\to\infty$ is still satisfied since both normal distributions are raised to strictly negative exponent dependent on $x$.
\section*{\textbf{(2)}}
\subsection*{\textbf{(a)}}
We want to show that 
\begin{equation}
\begin{aligned}
(\pdv[2]{}{t} - c^2\pdv[2]{}{x})u =\delta(x-\xi)\delta(t-\tau)\\
u = 0 \text{ , }\pdv{}{t}u = 0 \text{ at } t=0\\
\end{aligned}
\end{equation}
is equivalent to:
\begin{equation}
\begin{aligned}
(\pdv[2]{}{t} - c^2\pdv[2]{}{x})u = 0\\
u = 0 \text{ and } \pdv{}{t}u = \delta(x-\xi) \text{ at } t=\tau
\end{aligned}
\end{equation}
To do so, we make some observations: \\
\\
(1): On account of the time delta function, by definition the RHS of the non-homogeneous equation is zero at all $t$ except for $t=\tau$. Also, because the initial condition is $u=0$ and $\pdv{}{t}u = 0$ at $=0$, we know that nothing happens until $t=\tau$. \\
\\
(2): We also know that when $t>\tau$, the RHS is again zero because of the delta function. So the original equation is actually homogeneous everywhere except at $t=\tau$.\\
\\
(3): So the question is, what happens at $t=\tau$ in the non-homogeneous PDE? We need this information to define our second initial condition as well as to ascertain whether there is a jump in u at $t=\tau$, which would make this task impossible. To do this we integrate the non-homogeneous PDE with respect to time across an arbitrarily small domain around $t=\tau$:
\begin{equation}
\begin{aligned}
\int_{\tau -}^{\tau +}\Big{[} (\pdv[2]{}{t} - c^2\pdv[2]{}{x})u \Big{]}dt = \int_{\tau -}^{\tau+}\delta(x-\xi)\delta (t-\tau)dt\\
\int_{\tau -}^{\tau +}\pdv[2]{}{t}u dt - \int_{\tau -}^{\tau +}c^2\pdv[2]{}{x}u dt = \delta (x-\xi)\int_{\tau -}^{\tau+}\delta (t-\tau)dt
\end{aligned}
\end{equation}
Now a few observations. We know that when $t<\tau$ that $u=0$, so the first integral is simply equivalent to $\pdv{}{t}u$ evaluated at $\tau+$. Integrating the spatial second derivative across an arbitrarily small time span must be zero. If this weren't the case, $u$ itself would be a delta function in time, and the second derivative of $u$ in the original PDE doesn't indicate any huge singularity like that on the RHS. Lastly, the RHS is equal to $\delta(x-\xi)$ since the derivative across the zero of a delta function is by definition equal to 1. This means the second BC is:
\begin{equation}
\begin{aligned}
\pdv{}{t} \Big{|}_{\tau +} = \delta(x-\xi)
\end{aligned}
\end{equation}
Lastly, we can observe that if there was a jump at $t=\tau$, $\pdv{}{t}u$ at that precise point would be very large, i.e. it would itself be equal to some kind of $t$ delta function. This isn't the case, and so we know there is no jump there and thus the other initial condition must be $u=0$ at $t=\tau$, and we have achieved our objective.

\end{document}