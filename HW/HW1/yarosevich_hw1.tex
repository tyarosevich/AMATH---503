\documentclass{article}
\usepackage{fancyhdr}
\pagestyle{fancy}
\fancyfoot{}
\fancyfoot[C]{\thepage}
\rhead{Yarosevich - 1064712}

\usepackage{siunitx} % Provides the \SI{}{} and \si{} command for typesetting SI units
\usepackage{graphicx} % Required for the inclusion of images
\usepackage{amsmath} % Required for some math elements 
\usepackage[export]{adjustbox} % loads also graphicx
\usepackage{listings}
\usepackage{setspace}
\usepackage{matlab-prettifier}
\usepackage{float}
\usepackage[most]{tcolorbox}
\usepackage{amsfonts}
\usepackage{color}
\usepackage{titlesec}
\usepackage{caption}
\usepackage{subcaption}
\usepackage{placeins}
\usepackage{bm}
\usepackage{esvect}
\newcommand{\uveci}{{\bm{\hat{\textnormal{\bfseries\i}}}}}
\newcommand{\uvecj}{{\bm{\hat{\textnormal{\bfseries\j}}}}}
\DeclareRobustCommand{\uvec}[1]{{%
  \ifcsname uvec#1\endcsname
     \csname uvec#1\endcsname
   \else
    \bm{\hat{\mathbf{#1}}}%
   \fi
}}


\newcommand{\R}{\mathbb{R}}

\usepackage{xcolor}

\DeclareCaptionFont{white}{\color{white}}
\DeclareCaptionFormat{listing}{%
  \parbox{\textwidth}{\colorbox{gray}{\parbox{\textwidth}{#1#2#3}}\vskip-4pt}}
\captionsetup[lstlisting]{format=listing,labelfont=white,textfont=white}
\lstset{frame=lrb,xleftmargin=\fboxsep,xrightmargin=-\fboxsep}
\titleformat{\section}[runin]
  {\normalfont\Large\bfseries}{\thesection}{1em}{}
\titleformat{\subsection}[runin]
  {\normalfont\large\bfseries}{\thesubsection}{1em}{}


\setlength\parindent{0pt} % Removes all indentation from paragraphs

\renewcommand{\labelenumi}{\alph{enumi}.} % Make numbering in the enumerate environment by letter rather than number (e.g. section 6)

%\usepackage{times} % Uncomment to use the Times New Roman font

%----------------------------------------------------------------------------------------
%	DOCUMENT INFORMATION
%----------------------------------------------------------------------------------------

\title{AMATH 503: Homework 1 \\Due April, 15 2019 \\ ID: 1064712} % Title

\author{Trent \textsc{Yarosevich}} % Author name

\date{\today} % Date for the report

\begin{document}
\maketitle % Insert the title, author and date
\setlength\parindent{1cm}

\begin{center}
\begin{tabular}{l r}
%Date Performed: December 1, 2017 \\ % Date the experiment was performed
Instructor: Ka-Kit Tung % Instructor/supervisor
\end{tabular}
\end{center}

% If you wish to include an abstract, uncomment the lines below
% \begin{abstract}
% Abstract text
% \end{abstract}

%----------------------------------------------------------------------------------------
%	SECTION 1
%----------------------------------------------------------------------------------------
\section*{\textbf{(1)}}
I'll begin by doing separation of variables to find a general solution with the following assumption:
\begin{equation}
\begin{aligned}
u_x = T(t)X'(x)\\
u_{xx} = T(t)X''(x)\\
u_t = T'(t)X(x)\\
u_{tt} = T''(t)X(x)\\
\end{aligned}
\end{equation}
Plugging this into the PDE and isolating terms based on variable we get:
\begin{equation}
\begin{aligned}
\frac{T''(t)}{c^2T(t)} = \frac{X''(x)}{X(x)} = K
\end{aligned}
\end{equation}
As we've learned in class, we know $K>0$ and $K=0$ will give trivial solutions, so let $K = -\lambda^2$, yielding two ODEs. Begining with the x-dependent ODE we have, by ansatz:
\begin{equation}
\begin{aligned}
\frac{X''(x)}{X(x)} = -\lambda^2\\
X''(x) + \lambda^2X(x) = 0\\
X(x) = Asin(\lambda x) + Bcos(\lambda x)\\
\end{aligned}
\end{equation}
Now applying the BCs:
\begin{equation}
\begin{aligned}
X(0) = Asin(0) + Bcos(0) = 0\\
B = 0\\
X(L) = Asin(\lambda L) = 0\\
\lambda L = n\pi \\
\lambda = \frac{n\pi}{L}\\
X(x) = X_n(x) = sin(\frac{n\pi x}{L})
\end{aligned}
\end{equation}
Moving on to the time dependent ODE:
\begin{equation}
\begin{aligned}
\frac{T''(t)}{c^2T(t)}= -\lambda^2\\
T''(t) + c^2\lambda^2 T(t) = 0\\
T(t) = Asin(\lambda ct) + Bcos(\lambda ct)\\
\text{Let $\omega_n = c\lambda_n$}\\
T(t) = T_n(t) = A_nsin(\omega_n t) + B_ncos(\omega_n t)
\end{aligned}
\end{equation}
Combining the two ODEs, absorbing the $A_n$ from the x-dependent ODE into the $A_n$ and $B_n$ from the t-dependeont one, and using the principle of superposition, we have a sum of possible linear combinations of different solutions:
\begin{tcolorbox}[minipage,colback=white,arc=0pt,outer arc=0pt]
\begin{equation}
\begin{aligned}
u(x,t) = \sum^{\infty}_{n=1}\big{(}A_nsin(\omega_n t) + B_ncos(\omega_n t)\big{)}sin(\frac{n\pi x}{L})
\end{aligned}
\end{equation}
\end{tcolorbox}
We now apply the initial conditions, so that we can determine the arbitrary constants, beginning with the IC: $u(x,0) = f(x)$.
\begin{equation}
\begin{aligned}
u(x,0) = \sum^{\infty}_{n=1}B_nsin(\frac{n\pi x}{L}) = f(x)
\end{aligned}
\end{equation}
This is a sine series, so we can use the formula derived in class and in the notes using the orthogonality of the basis to determine $B_n$:
\begin{equation}
\begin{aligned}
B_n = \frac{2}{L}\int_0^L f(x)sin\frac{n\pi x}{L}dx
\end{aligned}
\end{equation}
We render this in a piecewise integral given the $f(x)$ provided in the prompt:
\begin{equation}
\begin{aligned}
B_n = \frac{2}{L}\big{[}\int_0^L xsin\frac{n\pi x}{L}dx + \int_0^L (L-x)sin\frac{n\pi x}{L}dx\big{]}\\
B_n = \frac{2}{L}\big{[}\int_0^L xsin\frac{n\pi x}{L}dx + \int_0^L Lsin\frac{n\pi x}{L}dx -\int_0^L xsin\frac{n\pi x}{L}d \big{]}\\
\end{aligned}
\end{equation}
First I will use integration by parts to determine an indefinite integral of $xsin(\frac{n\pi x}{L})$.
\begin{equation}
\begin{aligned}
u = x, du = 1\\
dv = sin(\frac{n\pi x}{L})\\
v = -\frac{L}{n\pi}cos(\frac{n\pi x}{L})\\
uv - \int vdu = \frac{-Lx}{n\pi}cos\frac{n\pi x}{L} + \frac{L^2}{n^2\pi^2}sin\frac{n\pi x}{L}
\end{aligned}
\end{equation}
Substituting this into the formula for $B_n$ we have:
\begin{equation}
\begin{aligned}
B_n = \frac{2}{L}\big{[}[ \frac{-Lx}{n\pi}cos\frac{n\pi x}{L} + \frac{L^2}{n^2\pi^2}sin\frac{n\pi x}{L}]\Big{|}^{\frac{L}{2}}_0 -[\frac{cos(\frac{n\pi x}{L})}{n\pi}]\Big{|}^L_{\frac{L}{2}}\\
-[\frac{-Lx}{n\pi}cos\frac{n\pi x}{L} + \frac{L^2}{n^2\pi^2}sin\frac{n\pi x}{L}]\Big{|}^L_{\frac{L}{2}}\\
B_n = \frac{2}{L}\Big{[}\Big{(}\frac{L^2}{n^2\pi^2}sin(\frac{n\pi}{2})\Big{)} + \Big{(}\frac{-L^2(-1)^n}{n\pi}\Big{)} - \Big{(}(-1)^n(\frac{-L^2}{n\pi}) - \frac{L^2}{n^2\pi^2}sin(\frac{n\pi}{2})\Big{)}\Big{]}\\
B_n = \frac{2}{L}\Big{[}\frac{2L^2}{n^2\pi^2}sin(\frac{n\pi}{2})\Big{]}
\end{aligned}
\end{equation}
\begin{tcolorbox}[minipage,colback=white,arc=0pt,outer arc=0pt]
\begin{equation}
\begin{aligned}
B_n = \frac{4L}{n^2\pi^2}sin(\frac{n\pi}{2})
\end{aligned}
\end{equation}
\end{tcolorbox}
To apply the other initial condition we must note that we can move the differential operator inside the summation is a linear combination of well-behaved functions.
\begin{equation}
\begin{aligned}
u_t(x,t)= \frac{d}{dt}\Big{[}\sum^{\infty}_{n=1}\big{(}A_nsin(\omega_n t) + B_ncos(\omega_n t)\big{)}sin(\frac{n\pi x}{L})\Big{]}\\
u_t(x,t) = \sum^{\infty}_{n=1}A_n\omega_ncos(\omega_n t)sin(\frac{n\pi x}{L}) - B_n\omega_nsin(\omega_n t)sin(\frac{n\pi x}{L})\\
u_t(x,0) = \sum^{\infty}_{n=1}A_n\omega_nsin(\frac{n\pi x}{L})=0
\end{aligned}
\end{equation}
Here we can observe that in order to satisfy the initial condition across all values of $x$, the arbitrary constant $A_n$ must be equal to zero. Thus we can now plug the values $A_n$ and $B_n$ from \textbf{(12)} above and the general solution from \textbf{(6)} to obtain a solution that satisfies all the Boundary and Initial Conditions:
\begin{tcolorbox}[minipage,colback=white,arc=0pt,outer arc=0pt]
\begin{equation}
\begin{aligned}
u(x,t) = \frac{4L}{\pi^2}\sum^{\infty}_{n=1}\frac{sin(\frac{n\pi}{2})}{n^2}cos(\omega_n t)sin(\frac{n\pi x}{L})
\end{aligned}
\end{equation}
\end{tcolorbox}
\section*{\textbf{(2)}}
\subsection*{\textbf{(a)}}
If the time derivative is equal to zero, we have:
\begin{equation}
\begin{aligned}
\alpha^2u_{xx} - bu = 0\\
u_{xx} = \frac{b}{\alpha^2}u
\end{aligned}
\end{equation}
I simply noted by inspection that the solution must be an exponential with a positive or negative coefficient in front of $x$ equal to the inverse of $\sqrt{\frac{b}{\alpha^2}}$, thus when multiplied through twice by taking the derivative, we satisfy the PDE. Any linear combination of such terms works, thus:
\begin{equation}
\begin{aligned}
u(x) = c_1e^{\frac{\sqrt{b}}{\alpha}x} + c_2e^{-\frac{\sqrt{b}}{\alpha}x}
\end{aligned}
\end{equation}
We then apply the boundary conditions:
\begin{equation}
\begin{aligned}
u(0) = c_1 + c_2 = 0\\
c_1 = -c_2\\
u(L) = c_1e^{\frac{\sqrt{b}}{\alpha}L} -c_1e^{-\frac{\sqrt{b}}{\alpha}L} = 0\\
c_1e^{\frac{\sqrt{b}}{\alpha}L} =c_1e^{-\frac{\sqrt{b}}{\alpha}L} 
\end{aligned}
\end{equation}
Because the two exponentials are different, the only way to satisfy this equation is if $c_1 = c_2 = 0$, which is perfectly sensible since we would expect a rod that is dissipating heat to eventually reach a zero temperature.
\subsection*{\textbf{(b)}}
We proceed by separation of variables, assuming that the solution is of the form $u(x,t) = X(x)T(t)$. \textbf{**Note} that I will reclaim the results from \textbf{(1)} and \textbf{(3) / (4)} above. Substituting these into the PDE gives:
\begin{equation}
\begin{aligned}
\frac{T''(t)}{T(t)\alpha^2}+\frac{b}{\alpha^2} = \frac{X''(x)}{X(x)} = K 
\end{aligned}
\end{equation}
Again let $K=-\lambda^2$ since we know other values give trivial solutions. We also already know the solution for $X(x)$ from above, thus we turn to the t-dependent ODE:
\begin{equation}
\begin{aligned}
\frac{T''(t)}{T(t)\alpha^2}+\frac{b}{\alpha^2} = -\lambda^2\\
T'(t) + bT(t) + \lambda^2\alpha^2T(t) = 0\\
T'(t) + (\lambda^2\alpha^2+b)T(t) = 0\\
T'(t) = -(\lambda^2\alpha^2 + b)T(t)
\end{aligned}
\end{equation}
Now let $c=-(\lambda^2\alpha^2 + b)$ and we have an easily solved ODE using an exponential:
\begin{equation}
\begin{aligned}
T(t) = \frac{1}{c}e^{ct}\\
T_n(t) = A_ne^{ct}\\
n = 1,2,3...
\end{aligned}
\end{equation}
Utilizing this $T_n(t)$ and the result from part one for the $X(x)$ term and combining arbitrary constants, we thus have an infinite number of solutions given by $u(x,t) = X_n(x)T_n(t)$:
\begin{equation}
\begin{aligned}
u(x,t) = \sum_{n=1}^\infty A_ne^{-(\alpha^2\lambda^2 + b)t}sin(\frac{n\pi x}{L})
\end{aligned}
\end{equation}
As in sections \textbf{(3) / (4)} above, this already satisfies the BCs since they are the same in part 2. The initial condition is given by:
\begin{equation}
\begin{aligned}
u(x,0) = \sum_{n=1}^\infty A_nsin(\frac{n\pi x}{L})
\end{aligned}
\end{equation}
This is a sine series and its coefficient can be determined for any $f(x)$ in the domain $0<x<L$ with the formula derived in class and given in the notes:
\begin{equation}
\begin{aligned}
A_n = \frac{2}{L}\int_0^Lf(x)sin(\frac{n\pi x}{L})
\end{aligned}
\end{equation}
Thus our solution satisfies the PDE, IC and BCs. To finally answer the prompt, we take the limit as $t\to\infty$:
\begin{tcolorbox}[minipage,colback=white,arc=0pt,outer arc=0pt]
\begin{equation}
\begin{aligned}
\lim_{t\to\infty}\sum_{n=1}^\infty A_ne^{-(\alpha^2\lambda^2 + b)t}sin(\frac{n\pi x}{L})
\end{aligned}
\end{equation}
We can see that the exponential will go to zero, and the steady state solution is zero as $t\to\infty$, which is the same as part \textbf{(a)}.
\end{tcolorbox}

\end{document}
