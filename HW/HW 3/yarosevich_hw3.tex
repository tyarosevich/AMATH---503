\documentclass{article}
\usepackage{fancyhdr}
\pagestyle{fancy}
\fancyfoot{}
\fancyfoot[C]{\thepage}
\rhead{Yarosevich - 1064712}
\usepackage{setspace}
\usepackage{siunitx} % Provides the \SI{}{} and \si{} command for typesetting SI units
\usepackage{graphicx} % Required for the inclusion of images
\usepackage{amsmath} % Required for some math elements 
\usepackage[export]{adjustbox} % loads also graphicx
\usepackage{listings}
\usepackage{setspace}
\usepackage{matlab-prettifier}
\usepackage{float}
\usepackage[most]{tcolorbox}
\usepackage{amsfonts}
\usepackage{color}
\usepackage{titlesec}
\usepackage{caption}
\usepackage{subcaption}
\usepackage{placeins}
\usepackage{bm}
\usepackage{esvect}
\newcommand{\uveci}{{\bm{\hat{\textnormal{\bfseries\i}}}}}
\newcommand{\uvecj}{{\bm{\hat{\textnormal{\bfseries\j}}}}}
\DeclareRobustCommand{\uvec}[1]{{%
  \ifcsname uvec#1\endcsname
     \csname uvec#1\endcsname
   \else
    \bm{\hat{\mathbf{#1}}}%
   \fi
}}


\newcommand{\R}{\mathbb{R}}

\usepackage{xcolor}

\DeclareCaptionFont{white}{\color{white}}
\DeclareCaptionFormat{listing}{%
  \parbox{\textwidth}{\colorbox{gray}{\parbox{\textwidth}{#1#2#3}}\vskip-4pt}}
\captionsetup[lstlisting]{format=listing,labelfont=white,textfont=white}
\lstset{frame=lrb,xleftmargin=\fboxsep,xrightmargin=-\fboxsep}
\titleformat{\section}[runin]
  {\normalfont\Large\bfseries}{\thesection}{1em}{}
\titleformat{\subsection}[runin]
  {\normalfont\large\bfseries}{\thesubsection}{1em}{}


\setlength\parindent{0pt} % Removes all indentation from paragraphs

\renewcommand{\labelenumi}{\alph{enumi}.} % Make numbering in the enumerate environment by letter rather than number (e.g. section 6)

%\usepackage{times} % Uncomment to use the Times New Roman font

%----------------------------------------------------------------------------------------
%	DOCUMENT INFORMATION
%----------------------------------------------------------------------------------------

\title{AMATH 503: Homework 3 \\Due April, 29 2019 \\ ID: 1064712} % Title

\author{Trent \textsc{Yarosevich}} % Author name

\date{\today} % Date for the report

\begin{document}
\maketitle % Insert the title, author and date
\setlength\parindent{1cm}

\begin{center}
\begin{tabular}{l r}
%Date Performed: December 1, 2017 \\ % Date the experiment was performed
Instructor: Ka-Kit Tung % Instructor/supervisor
\end{tabular}
\end{center}
\doublespacing
% If you wish to include an abstract, uncomment the lines below
% \begin{abstract}
% Abstract text
% \end{abstract}

%----------------------------------------------------------------------------------------
%	SECTION 1
%----------------------------------------------------------------------------------------
\section*{\textbf{(1)}}
\subsection*{\textbf{(a)}}
Using the formula for a complex Fourier series in the notes we have the following in a domain of $-2< x < 2$ with $n=0, \pm 1, \pm 2, \pm 3...$
\begin{equation}
\begin{aligned}
|x| = \sum_{-\infty}^{\infty}c_ne^{\frac{-in\pi x}{2}}
\end{aligned}
\end{equation}
Similarly, we obtain the coefficient $c_n$ with the following:
\begin{equation}
\begin{aligned}
c_n = \frac{1}{4}\int_{-2}^2 |x|e^{\frac{in\pi x}{2}}dx
\end{aligned}
\end{equation}
To solve this, we split the integral into two domains and integration by parts:
\begin{equation}
\begin{aligned}
c_n = \frac{1}{4}\Big{[}\int_{0}^2 xe^{\frac{in\pi x}{2}}dx - \int_{-2}^{0} xe^{\frac{in\pi x}{2}}dx \Big{]}\\
u = x \text{ , } du = 1\\
dv = e^{\frac{in\pi x}{2}}\\
v = \frac{2}{in\pi}e^{\frac{in\pi x}{2}}\\
uv = \int vdu = \frac{2x}{in\pi}e^{\frac{in\pi x}{2}} - \frac{4}{n^2\pi^2}e^{\frac{in\pi x}{2}}\\
c_n = \frac{1}{4}\Bigg{[}\Big{[}\frac{2xe^{\frac{in\pi x}{2}}}{in\pi} + \frac{4e^{\frac{in\pi x}{2}}}{n^2\pi^2}\Big{]}|^2_0 - \Big{[}\frac{2xe^{\frac{in\pi x}{2}}}{in\pi} + \frac{4e^{\frac{in\pi x}{2}}}{n^2\pi^2}\Big{]}|^0_{-2}\Bigg{]}\\
c_n = \frac{1}{4}\Bigg{[}\Big{[}\frac{4e^{in\pi}}{in\pi} + \frac{4e^{in\pi}}{n^2\pi^2} - \frac{4}{n^2\pi^2}\Big{]} - \Big{[}\frac{4}{n^2\pi^2}+\frac{4e^{-in\pi}}{in\pi} + \frac{4e^{-in\pi}}{n^2\pi^2}\Big{]}\Bigg{]}\\
c_n = \frac{-in\pi e^{in\pi}+in\pi e^{-in\pi} + e^{in\pi} + e^{-in\pi} - 2}{n^2\pi^2}
\end{aligned}
\end{equation}
Now we observe that:
\begin{equation}
\begin{aligned}
e^{in\pi} = cos(n\pi) + isin(n\pi) = cos(n\pi) = (-1)^n\\
e^{-in\pi} = cos(n\pi) - isin(n\pi) = cos(n\pi) = (-1)^n\\
e^{in\pi} = e^{-in\pi}
\end{aligned}
\end{equation}
This allows for the following simplification:
\begin{equation}
\begin{aligned}
c_n = \frac{2(-1)^n - 2}{n^2\pi^2}
\end{aligned}
\end{equation}
This is obviously not defined for $n=0$ however, and so we calculate this independently:
\begin{equation}
\begin{aligned}
c_0 = \frac{1}{4}\int_{-2}^2|x|dx\\
c_0 = \frac{1}{4} \Big{[}[\frac{x^2}{2}]\Big{|}_0^2 - [\frac{x^2}{2}]\Big{|}_{-2}^0 \Big{]} = 1
\end{aligned}
\end{equation}
Putting all this togther, we arrive at the following Fourier series, noting that the even modes are zero (except $c_0$), and that $n = \pm 1, \pm 2, \pm 3...$
\begin{tcolorbox}[minipage,colback=white,arc=0pt,outer arc=0pt]
\begin{equation}
\begin{aligned}
|x| = 1 + \frac{2}{\pi^2}\sum_{n=-\infty}^{\infty}\frac{(-1)^n - 1}{n^2}e^{-\frac{in\pi x}{2}}
\end{aligned}
\end{equation}
\end{tcolorbox}
\subsection*{\textbf{(b)}}
Using Euler's Formula we can rewrite the exponential above in the following way:
\begin{equation}
\begin{aligned}
e^{-\frac{in\pi x}{2}} = cos(\frac{n\pi x}{2}) - isin(\frac{n\pi x}{2})
\end{aligned}
\end{equation}
If we observe that $sin$ is an odd function, it follows that $isin(\frac{-n\pi x}{2}) = - isin(\frac{n\pi x}{2})$. Additionally, the $n$ term in the divisor is squared, so it is no different for $\pm n$. This means that when $n<0$, all the $sin$ values in the sum will cancel with their corresponding values from $n>0$, for example when $n=1$ we have  $- isin(\frac{\pi x}{2})$ and when $n=-1$ we have   $isin(\frac{n\pi x}{2})$.\\
Similarly, we can observe that cosine is an even function and $cos(nx) = cos(-nx)$. Thus we can change our sum to be from $0< n < \infty$ and multiply the sum by 2, noting that $n=0$ is accounted for already from c_0.
\\
Lastly, we can also observe that the even modes are zero, substituting $n = 2k-1$ and define our sum with $k = 1, 2, 3...$. This means $c_n = \frac{-4}{n^2\pi^2}$ as well. The resulting cosine series after discarding the sines, changing the range of $n$, eliminating even modes, and multiplying by 2, is:
\begin{tcolorbox}[minipage,colback=white,arc=0pt,outer arc=0pt]
\begin{equation}
\begin{aligned}
|x| = 1 - \frac{8}{\pi^2}\sum_{k=1}^{\infty}\frac{cos(\frac{(2k-1)\pi x}{2})}{(2k-1)^2}
\end{aligned}
\end{equation}
\end{tcolorbox}
\subsection*{\textbf{(c)}}
\begin{equation}
\begin{aligned}
\frac{d}{dx}[1 - \frac{8}{\pi^2}\sum_{k=1}^{\infty}\frac{cos(\frac{(2k-1)\pi x}{2})}{(2k-1)^2}] = \frac{4}{\pi}\sum_{k=1}^{\infty}\frac{sin(\frac{n\pi x}{2})}{2k-1}
\end{aligned}
\end{equation}
This is the sine series for $f(x) =1$ in a domain of $0<x<2$. To show this, we simply observe that if $f(x) = 1$, our sine series coefficient is:
\begin{equation}
 \begin{aligned}
 a_n = \int_0^2 sin(\frac{n\pi x}{2})\\
 a_n = \Big{[}-\frac{2}{n\pi}cos(\frac{n\pi x}{2}) \Big{]}\Big{|}_0^2\\
 a_n = \frac{2 - 2(-1)^n}{n\pi}\\
 \text{Even modes are zero, thus:}\\
 f(x) = \sum_{n=1}^{\infty}\frac{4}{n\pi}sin(\frac{n\pi x}{2})\\
 f(x) = \frac{4}{\pi}\sum_{k=1}^{\infty}\frac{sin(\frac{(2k-1)\pi x}{2})}{2k-1}
 \end{aligned}
 \end{equation} 

This is the same as the result of taking our derivative, which makes sense since we know the derivative of $|x|$ is -1 from $-\infty < x < 0$ and 1 from $0<x<\infty$. Furthermore, we know that a fourier sine series for a non-periodic function will only be accurate in the domain $0 < x< L$, and because it's an odd function, will be reflected accordingly outside that domain. Thus we would expect the Fourier series for the derivative of $|x|$ to be accurate from $-L < x < L$ and be $2L$ periodic outside this. The resulting function would be the sine series of $f(x) = 1$ and that is precisely what we got.
\section*{\textbf{(2)}}
We first define the fourier transforms in the different dimensions:
\begin{equation}
\begin{aligned}
\mathcal{F}_1[u(x,y,z,t)] = \int_{-\infty}^\infty u(x,y,z,t,)e^{i\omega_1 x}dx\\
\mathcal{F}_2[u(x,y,z,t)] = \int_{-\infty}^\infty u(x,y,z,t,)e^{i\omega_2 y}dy\\
\mathcal{F}_3[u(x,y,z,t)] = \int_{-\infty}^\infty u(x,y,z,t,)e^{i\omega_3 z}dz\\
\end{aligned}
\end{equation}
We can then fully define $u$ in fourier space:
\begin{equation}
\begin{aligned}
\mathcal{F}_1[\mathcal{F}_2[\mathcal{F}_3[u(x,y,z,t)]]] = U(\omega_1, \omega_2, \omega_3, t)
\end{aligned}
\end{equation}
Now defining the terms in the PDE, in particular using the definition of a derivative in Fourier space from the notes:
\begin{equation}
\begin{aligned}
\mathcal{F}_1\mathcal{F}_2\mathcal{F}_3[u_t] = \frac{d}{dt}U(\omega_1, \omega_2, \omega_3, t)\\
\mathcal{F}_1\mathcal{F}_2\mathcal{F}_3[u_xx] = (i\omega_1)^2 U(\omega_1, \omega_2, \omega_3, t)\\
\mathcal{F}_1\mathcal{F}_2\mathcal{F}_3[u_yy] = (i\omega_2)^2 U(\omega_1, \omega_2, \omega_3, t)\\
\mathcal{F}_1\mathcal{F}_2\mathcal{F}_3[u_zz] = (i\omega_3)^2 U(\omega_1, \omega_2, \omega_3, t)\\
\end{aligned}
\end{equation}
We can now write out the full PDE in Fourier space:
\begin{equation}
\begin{aligned}
\frac{d}{dt}U = -D(\omega_1^2 + \omega_2^2 + \omega_3^2)U
\end{aligned}
\end{equation}
We then separate variables and integrate with respect to $t$, letting $D = \alpha^2$:
\begin{equation}
\begin{aligned}
\frac{dU}{U} = -\alpha^2(\omega_1^2 + \omega_2^2 + \omega_3^2)dt\\
lnU = -\alpha^2(\omega_1^2 + \omega_2^2 + \omega_3^2)t+ C\\
U = Ae^{-\alpha^2(\omega_1^2 + \omega_2^2 + \omega_3^2)t}
\end{aligned}
\end{equation}
Applying the initial condition, we have:
\begin{equation}
\begin{aligned}
A = U(\omega_1, \omega_2, \omega_3, 0)\\
U(\omega_1, \omega_2, \omega_3, 0) = \mathcal{F}_1\mathcal{F}_2\mathcal{F}_3\delta (x)\delta (y) \delta(z)
\end{aligned}
\end{equation}
As we learned in class, the integral across a delta function is equal to 1 because it is summing the area under an infinitely narrow Riemann rectangle with infinite height. Thus we have:
\begin{equation}
\begin{aligned}
U = e^{-\alpha^2(\omega_1^2 + \omega_2^2 + \omega_3^2)t}\\
\end{aligned}
\end{equation}
To apply the reverse transform, we can split $U$ up into three exponentials and apply the reverse transform to each one:
\begin{equation}
\begin{aligned}
U = e^{-\alpha^2\omega_1^2t}e^{-\alpha^2\omega_2^2t} e^{-\alpha^2\omega_3^2t}\\
u(x,y,z,t) &= \frac{1}{2\pi}\int_{-\infty}^{\infty}e^{-\alpha^2\omega_1^2t}e^{i\omega_1 x}d\omega_1\\
&*\frac{1}{2\pi}\int_{-\infty}^{\infty}e^{-\alpha^2\omega_2^2t}e^{i\omega_1 y}d\omega_2\\
&*\frac{1}{2\pi}\int_{-\infty}^{\infty}e^{-\alpha^2\omega_3^2t}e^{i\omega_1 z}d\omega_3
\end{aligned}
\end{equation}
These integrals can be re-written in the form:
\begin{equation}
\begin{aligned}
\frac{1}{2\pi}\int_{-\infty}^{\infty}e^{-\alpha^2\omega_1^2t + i\omega_1 x}d\omega_1
\end{aligned}
\end{equation}
Using the general formula for Euler's Integral, taken from the class notes section (8.2), and noting that $t$ is being treated as a constant here since we are integrating in respect to the omegas. The formula is as follows:
\begin{equation}
\begin{aligned}
\int_{-\infty}^{\infty}e^{p^2\omega^2 \pm q\omega}d\omega = \frac{\sqrt{\pi}}{p}e^{\frac{q^2}{4p^2}}
\end{aligned}
\end{equation}
Applying this to each integral, we get the answer:
\begin{tcolorbox}[minipage,colback=white,arc=0pt,outer arc=0pt]
\begin{equation}
\begin{aligned}
u(x,y,z,t) = (\frac{1}{2\pi})^3(\frac{\sqrt{\pi}}{\alpha\sqrt{t}})^3e^{\frac{-x^2}{4\alpha^2 t}}e^{\frac{-y^2}{4\alpha^2 t}}e^{\frac{-z^2}{4\alpha^2 t}}\\
u(x,y,z,t) = \frac{1}{8\pi^3})(\frac{\pi^{\frac{3}{2}}}{\alpha^3t^{\frac{3}{2}}})e^{-\frac{(x^2+y^2+z^2)}{4\alpha^2 t}}\\
u(x,y,z,t) = \frac{1}{8\alpha^3 t^{\frac{3}{2}}\pi^{\frac{3}{2}}}e^{-\frac{(x^2+y^2+z^2)}{4\alpha^2 t}}\\
\end{aligned}
\end{equation}
\end{tcolorbox}

\end{document}
